\section{Semantics}
In this section, we describe the behaviour of all types and operations in \tetra.
First, we describe the notation for various operations and their meaning.
Afterwards, we create a calculus which defines all operations formally and allows the proving of behaviour.

\subsection{Informal\optional}

\subsubsection{Expressions}
If $\cn o$ is an object, then let $\cn o$ denote a reference to $\cn o$, and $\val{\cn o}$ refer to $\cn o$'s value.
Thus, for an object
	$\var{o}{\struct{\var{a}{\bool}}{\var{b}{\N_8}}}$,
	$\memb{\cn o}{a} \in \reft\bool$ is a reference to the field $\cn a$ in $\cn o$, while
	$\val{\memb{\cn o}{a}} \in \bool$ refers to the value of $\cn a$ in $\cn o$.
If $v,a,b$ are values with $v \in \bool$, then let
	$\ifexp{v}{a}{b}$
	evaluate to $a$ if $v = \true$, otherwise $b$.

\subsubsection{Statements}
\paragraph{Parallel and sequential statements}
Let $s_1, s_2$ be operations.
Then let $\parstmt{s_1}{s_2}$ be $s_1$ and $s_2$ in parallel, and
	$\stmt{s_1}{s_2}$ be $s_1$ and $s_2$ in sequence.
Parallel statements are executed simultaneously.
Sequential statements are executed in the order in which they are written.

\paragraph{Assignments}
Let $v$ be a reference value, and $x$ a value, then let $\stmt{v \gets x}$ be an assignment statement of $x$ into $v$.

\paragraph{Function calls}
Let $f$ be a function, then let $s = \stmt{f(a,b,c)}$ be a statement invoking $f$ with the arguments $a$, $b$, and $c$.
This is equivalent to a replacement of $s$ with $f$'s definition, and initialising $f$'s parameters to to $a$, $b$, and $c$, respectively.

\subsection{Formal}

Let $\State = \Symbol \to \Value$, where $\Symbol$ is the set of all variable names, and $\Value$ is the set of all object values be the set of all possible execution states.
$\State$ is thus a mapping of variable names to variable values.
Furthermore, let $\varnothing \in \State \cap \Value$ be the value for non-existant variables.
Let $\dagger$ be a special value reserved for destroyed variables.
Let $\Box$ be a special value for existing, but inaccessible (\emph{blocked}) variables.
Then $\varnothing(k): \Symbol \to \Value = \varnothing$ is the empty state.
For $s \in \State$, let
$$
\StSet{s}{k}{v}: \Symbol \times \Value \to \State =
	\kappa \mapsto \begin{cases}
		v & \text{if } \kappa = k,\\
		s(\kappa)
\end{cases}$$
and
$$
\StDel{s}{k}: \Symbol \to \State = \kappa \mapsto \begin{cases}
	\varnothing & \text{if } \kappa = k,\\
	s(\kappa)
\end{cases}
$$
be the functions to (re-)define and delete a variable in a state, respectively.

\paragraph{State operations}
For $\sigma_1, \sigma_2 \in \State$, let
$$\text{(Merging) }
	\sigma_1 \cup \sigma_2 =
		\kappa \mapsto \begin{cases}
			\sigma_2(\kappa) &\text{if } \sigma_2(\kappa) \notin \set{\varnothing, \Box},\\
			\sigma_1(\kappa)
		\end{cases}
$$
$$\text{(Blocking) }
	\sigma_1 \setminus \sigma_2 =
		\kappa \mapsto \begin{cases}
			\sigma_1(\kappa) &\text{if } \sigma_2(\kappa) \in \set{\varnothing, \Box},\\
			\dagger &\text{if } \sigma_2(\kappa) = \dagger,\\
			\Box
		\end{cases}
$$
Note that both operations are not commutative.
When merging to states using $\cup$, the right-hand state is the ``newer'' or ``more recent'' state and its values take precedence.
When blocking a state's values in another state using $\setminus$, all existing values in the right-hand state become blocked in the left-hand state.

\paragraph{Values}
All $v \in \Value$ are of the form $v = (i, T)$ where $i$ is the internal representation of the value and $T$ is its type.
Then $\tau(v) = T$ resolves the type and $\psi(v) = i$ the internal representation of the given value.
Let $a, b \in \Value$ be values, following is a general short-hand notation used throughout the whole paper: operations on values act on internal representation;
using the expression $a + b$ is equivalent to $\psi(a) + \psi(b)$ and the infix notation can be used instead of the prefix notation ($a + b \equiv \opexp{+}{a}{b}$).
Numbers of the unbounded integer type are denoted as $1^*, 2^*$, etc.

\subsubsection{Transition rules}
Let $\Sigma$ be a state.
This section describes the semantics for operations on $\Sigma$.
In the following, semantic rules are of the form
	$$\semrule{\mathit{Condition}}{{\Sigma, \mathit{Operation}}}{\Sigma'}$$
	where
		$\mathit{Operation}$ is the operation whose semantics is defined,
		$\mathit{Condition}$ is the condition under which the rule is valid, and
		$\Sigma'$ is the new state after $\mathit{Operation}$ is applied to $\Sigma$.
Thus, we describe big step semantics.
There can be multiple rules with different conditions for the same operation.
If an operation fulfills no conditions in any rules, it is invalid.
The function $\Delta(\Sigma, \mathit{Operation}) = \Sigma_\Delta$ returns a state containing only the variables affected by the operation.
Furthermore, we need the following relation:
$$\results{\Sigma}{\omega}{\Sigma'} = \begin{cases}
	\true & \text{if $\Sigma'$ is the result of operation $\omega$ on $\Sigma$,} \\
	\false & \text{otherwise.}
\end{cases}
$$
Another possible form is
$$
\semruleeq
	{\mathit{Condition}}
	{\Sigma, \mathit{Operation}}
	{\Sigma, \mathit{Operation'}}
$$
which states that under a certain condition, one operation can be replaced with a different, equivalent operation.

\paragraph{Expressions}
Let $\Expr: \State \to \State \times \Value$ be the set of all expressions.
An expression takes the program state and returns a modified state and result.
The function $\Delta(\Sigma, \Expr) = \Sigma_{\Delta}$ contains all variables affected by the expression.

\paragraph{Sequence reduction}
For any sequence of $n$ operations $\omega_1, \ldots, \omega_n$ the operations can be evaluated step by step.
$$
\semrule{
	\forall\, k \in \range{1}{n-1}: \results{\Sigma_k}{\stmt{\omega_k}}{\Sigma_{k+1}}}
	{\Sigma_1, \stmt{\omega_1}{\ldots}{\omega_n}}
	{\Sigma_n},
	\Sigma_1, \ldots, \Sigma_n \in \State
$$
$$
	\Delta(\Sigma_1, \stmt{\omega_1}{\ldots}{\omega_n}) =
		\Delta(\Sigma_1, \omega_1) \cup \ldots \cup \Delta(\Sigma_n, \omega_n)
$$

$$
\semrule{
	\forall\, k \in \range{1}{i-1}: \results{\Sigma_k}{\stmt{\omega_k}}{\Sigma_{k+1}}}
	{\Sigma_1, \stmt{\omega_1}{\ldots}{\omega_i}{\retstmt{x}}{\omega_{i+1}}{\ldots}{\omega_n}}
	{\Sigma_i},
	\Sigma_1, \ldots, \Sigma_i \in \State
$$
$$
	\Delta(\Sigma_1, \stmt{\omega_1}{\ldots}{\omega_i}{\retstmt{x}}{\omega_{i+1}}{\ldots}{\omega_n}) =
		\Delta(\Sigma_1, \omega_1) \cup \ldots \cup \Delta(\Sigma_i, \omega_i)
$$

$$
\semruleeq{\true}
	{\Sigma, \stmt{\omega_1}}
	{\Sigma, \omega_1}
$$

\paragraph{Parallel reduction}
For a parallel series of $n$ operations $\omega_1, \ldots, \omega_n$, their resulting state is merged using $\cup$.
No variable modified by one statement may be accessed in another statement.
$$
	\semrule{
		\results{\Sigma \setminus \Delta(\Sigma, \omega_2)}{\omega_1}{\Sigma_1} \land
		\results{\Sigma \setminus \Delta(\Sigma, \omega_1)}{\omega_2}{\Sigma_2}}
		{\Sigma, \parstmt{\omega_1}{\omega_2}}
		{\Sigma_1 \cup \Sigma_2},
			\Sigma_1, \Sigma_2 \in \State
$$
$$
	\Delta(\Sigma, \parstmt{\omega_1}{\ldots}{\omega_n}) = \Delta(\Sigma, \omega_1) \cup \ldots \cup \Delta(\Sigma, \omega_n)
$$
$$
	\semruleeq{\true}
		{\Sigma, \parstmt{\omega_1}{\omega_2}{\ldots}{\omega_n}}
		{\Sigma, \parstmt{\parstmt{\omega_1}{\omega_2}}{\omega_3}{\ldots}{\omega_n}}
$$
$$
\semruleeq{\true}
	{\Sigma, \parstmt{\omega_1}}
	{\Sigma, \omega_1}
$$

\paragraph{Variable declaration}
Declaring a variable creates an uninitialised entry for that variable in the state.
For all types $T$, let $T_\varnothing$ be the uninitialised value of type $T$.
Then a variable declaration works as follows:
$$
\semrule
	{\Sigma(x) = \varnothing}
	{\Sigma, \stmt{\varv{x}{T}}}
	{\StSet{\Sigma}{x}{T_\varnothing}},
		x \in \Symbol
$$
$$
\Delta(\Sigma, \stmt{\varv{x}{T}}) = \StSet{\varnothing}{x}{T_\varnothing},
                x \in \Symbol
$$

\paragraph{Variable assignment}
An assignment assigns a value to a variable.
This value has to be resolved already into a value of the same type as the variable.
Only existing variables can be assigned.
Let $\typeof{x}$ denote the type of $x$, where $x \in \Value$, and let $v \in \Symbol$.
For non-reference types, assignments are as follows:
$$
\semrule
	{\Sigma(v), x \notin \set{\varnothing, \dagger, \Box} \land
		\typeof{x} = \typeof{\Sigma(v)}}
	{\Sigma, \stmt{v \gets x}}
	{\StSet{\Sigma}{v}{x}},
$$
$$
\Delta(\Sigma, \stmt{v \gets x}) = \StSet{\varnothing}{v}{x}
$$
For any type $T$, reference assignments are as follows:
$$
\semrule
	{\Sigma(v_1) = \reft{T}_\varnothing \land
		\typeof{\Sigma(\Sigma(v_2))} = T}
	{\Sigma, \stmt{v_1 \gets v_2}}
	{\StSet{\Sigma}{v_1}{\Sigma(v_2)}}
$$

\paragraph{Variable destruction}
The destructor operation makes a variable invalid and no longer modifiable.
$$
\semrule
	{\Sigma(v) \notin \set{\varnothing, \dagger, \Box}}
	{\Sigma, \stmt{\dagger v}}
	{\StSet{\Sigma}{v}{\dagger}},
		v \in \Symbol
$$
$$
\Delta(\Sigma, \stmt{\dagger v}) = \StSet{\varnothing}{v}{\dagger}
$$


\paragraph{If statement}
Let $c$ be a resolved value, and let $\omega_1, \omega_2$ be two operations.
$$
\semruleeq
	{c = \true}
	{\Sigma, \stmt{\ifstmt{c}{\omega_1}{\omega_2}}}
	{\Sigma, \stmt{\omega_1}}
$$
$$
\semruleeq
	{c = \false}
	{\Sigma, \stmt{\ifstmt{c}{\omega_1}{\omega_2}}}
	{\Sigma, \stmt{\omega_2}}
$$
$$
\Delta(\Sigma, \ifstmt{c}{\omega_1}{\omega_2}) =
\begin{cases}
  \Delta(\Sigma', \stmt{c}{\omega_1}), ~ \text{if } c(\Sigma) = (\Sigma', \true)\\
  \Delta(\Sigma', \stmt{c}{\omega_2}), ~ \text{if } c(\Sigma) = (\Sigma', \false)
\end{cases}
$$

\paragraph{While statement}
Let $c$ be an expression, and let $\omega$ be an operation.
$$
\semruleeq
       {\true}
       {\Sigma, \whilestmt{c}{\omega}}
       {\Sigma, \ifstmt{c}{\stmt{\omega}{\whilestmt{c}{\omega}}}{\stmt}}
$$

\paragraph{For statement}
Let $c$ be an expression, and let $\omega_{1,\ldots,3}$ be operations.
$$
\semruleeq
        {\true}
        {\Sigma, \forstmt{\omega_1}{c}{\omega_2}{\omega_3}}
        {\Sigma, \stmt{\omega_1}{\whilestmt{c}{\stmt{\omega_3}{\omega_2}}}}
$$

\paragraph{Parallel for statement}
Let $c$ be an expression, and let operation $\omega(i)$ be the iteration body, where $i$ is the iterator value.
$$
\semruleeq
        {\true}
        {\Sigma, \parforstmt{v}{\omega}}
        {\Sigma, \parstmt
        	{\omega(a)}
        	{\omega(a+1)}
        	{\ldots}
        	{\omega(b)}},
        (\Sigma, a) = \Sigma(\val{\memb{v}{start}}),
        (\Sigma, b) = \Sigma(\val{\memb{v}{end}})
$$

\paragraph{Variable expression}
Let $v$ be a variable name.
$$
\Sigma(\val{v}) = (\Sigma, \Sigma(v)),
	\Sigma(v) \notin \set{\Box, \dagger, \varnothing}
$$
$$
\Delta(\Sigma, \val{v}) = \varnothing
$$

\paragraph{Function call expressions}
Let $f \in \Value^n \to \Operation \cup \Expr$ be an $n$-ary function, and $v_{1, \ldots, n}$ be values.
Then the invocation $f(v_1, \ldots, v_n)$ expands to the operation that defines the function.
Let $a_{1, \ldots, n}$ be expressions.
Then $\opexp{f}{a_1}{\ldots}{a_n}$ is a call expression.
$$
\Sigma(\opexp{f}{a_1}{\ldots}{a_n}) = \Sigma_n(\ttlparen f(v_1, \ldots, v_n) \ttrparen),
	(\Sigma_i, v_i) = \Sigma_{i-1}(a_i),
		\Sigma_0 = \Sigma, 1 \le i \le n
$$
\begin{equation*}
\begin{split}
\Delta(\Sigma, \opexp{f}{a_1}{\ldots}{a_n}) &=
	\Delta(\Sigma_n, \ttlparen f(v_1, \ldots, v_n) \ttrparen) ~ \cup ~ \cup_{1 \le i \le n} \Delta(\Sigma_i, a_i), \\
	&(\Sigma_i, v_i) = \Sigma_{i-1}(a_i), \Sigma_0 = \Sigma, 1 \le i \le n
\end{split}
\end{equation*}

\paragraph{Return statement}
Let $x$ be an expression.
Then $\retstmt{x}$ is a return statement and will terminate a sequence of operations early with the given expression as its result.
$$
\Sigma(\ttlparen \omega \ttrparen) = \Sigma'(\ttlparen \omega' \ttrparen),
	\equivs{\Sigma}{\omega}{\Sigma', \omega'}
$$
$$
\Sigma(\ttlparen \omega \ttrparen) = \Sigma',
	\results{\Sigma}{\omega}{\Sigma'}
$$
$$
\Delta(\Sigma, \ttlparen \omega \ttrparen) = \Delta(\Sigma, \omega)
$$
$$
\Sigma(\ttlparen \stmt{\retstmt{x}}{\ldots} \ttrparen) = (\Sigma', v) = \Sigma(x)
$$
$$
\Delta(\Sigma, \ttlparen \stmt{\retstmt{x}}{\ldots} \ttrparen) = \Delta(\Sigma, x)
$$
