\begin {center}

\vspace{1 in}
{\Huge\scshape{\tetra}\\}
{\Large~\\}
{\Large\scshape{A fresh way of describing computations\\}}
{\Huge~\\}
{\large\scshape{Norbert Dzikowski, Steffen Rattay\\
Draft of \today\\}}
{\Huge~\\}


\textit{
``An idiot admires complexity.
A genius admires simplicity.
A physicist tries to make it simple.
Anyway, an idiot -- anything, the more complicated it is -- the more he will admire it.
If you make something so cluster-fucked he can't understand it, he's gonna think you're a god, because you made it so complicated, nobody can understand it.
That's how they write academic journals, they try to make it so complicated, people think you're a genius.''}
--- {\scshape Terrence Andrew Davis}


\paragraph{Abstract}
In this document, we define the {\tetra} computation model's concepts and semantics.
The {\tetra} computation model provides the theoretical framework for the {\tetra} computing language, the syntax of which is not part of this work.
It is a general computation framework and can be used to describe abstract computations, programs, and the behaviour of digital circuits.

\end {center}
