\section{Values}
Let $\Value = \fund\Value \times \Type \times \TypeID$ be the set of all values.
A value has the form $(v, t, d)$, where $v$ is the actual value, $d$ is the dynamic/deriving type, and $t$ is the static type.
For non-polymorphic values, $d=t$, and for polymorphic values, $d$ is at least as specific as $t$.
\begin{itemize}
	\item Query for: What is the next lower level deriving type.
	\item Query for: What is the lowest level deriving type.
\end{itemize}

\section{Datatypes}

\subsection{Elementary types}
Let $\ElemType = \TypeID \times \fund\Type \times FunctionMap \times FieldMap \times P(\TypeID) \times P(\TypeID)$.
A type has the form $(i, t, fns, fields, b, d)$, where $i$ is the type's unique ID, $b$ is the set of the $\TypeID$s of its base types, and $d$ is the set of the $\TypeID$s of all types deriving from or implementing it.
Elementary types are formed from a descriptor, which can either be a structure or primitive descriptor:
$$
	(t, fields) = \DeclareStruct(\struct{\var{v_1}{T_1}, \dots, \var{v_n}{T_n}}) = (T_1 \times \dots \times T_n, (\cn{v_1}, \dots, \cn{v_n})),
$$
$$
	(t, ()) = \DeclarePrimitive(t)
$$
For structured types, the names of the fields and their types are extracted.
For primitive types, no fields exist, and only the type is needed.
An elementary type is then constructed from a descriptor $desc = (t, fields)$ as follows:
$$
	\DeclareElemType(id, desc, fns, b, d) = (id, t, fns, fields, b, d)
$$

\subsection{Arrays}
An array type is defined as $\ArrayType = \Type \times \N$.
An array has the form $(T, n)$, where $T$ is the element type and $n$ is the size of the array.
IDEA: arrays are a concept and not definite types, and any indexable type can be converted to an array type if the bounds match.

\begin{itemize}
	\item Classes/Structs:
\end{itemize}

\section{Statements}

\section{Expressions}
\subsection{Typecasts}
\begin{itemize}
	\item Conversion casts, e.g.: Int $\to$ Float.
	\item Determine the type of an expression based on context (outside-in) (Algorithm W).
	\item Model as function, which is called implicitly. Will be predefined for primitive types.
	\item Structural conversion casts using tuples: Detect conversion distance properly (sum over all element distances).
	\item Lens/inheritance casts, e.g.: InheritingClass $\to$ BaseClass $\to$ InheritingClass.
\end{itemize}

\section{Functions}
\begin{itemize}
	\item All functions have a unique type associated, in addition to their signature type.
		Type contains stack frame and general info (useful for coroutines, closures).
	\item Support new expression:
		\begin{itemize}
			\item Statements and expressions have to redefined on this level.
			\item Need to include casts, type-queries, etc.
		\end{itemize}
\end{itemize}

\section{Misc}
\begin{itemize}
	\item All type instances have to specified as either const or mut.
	\item VTables (TypeIDs + class descriptor objects).
	\item Everything is explicit, there are no implicit conversions.
	\item Introduce a conversion operator for different conversion kinds of cast (lens/inheritance casts, value conversion casts, \dots).
	\item Primitives ($\Z$, $\N$, \dots) are explicitly defined.
	\item When overriding member-functions, the function's base class has to be stated explicitly.
	\item Use outward-in type inferring for things like references etc. when instantiating a context.
\end{itemize}


\section{Memory management}
RAM is exposed as a compile-time builtin interface type (user-implementable).
RAM allows arbitrary read and write access to memory cells (also support read-only RAMs etc.).
\begin{itemize}
	\item References are user-defined types but with a native syntax to which meaning is bound by context.
	\item It is forbidden to get a valid reference to another object from an existing reference.
	\item (Native memory management): RAM is exposed as continuous byte array.
	\item There can be multiple RAMs.
	\item Context-local variables have different address types from RAM variables.
	\item The type of address a pointer points to is inferred by its possible usages.
	\item RAM modules and local variables are "reference factories".
\end{itemize}